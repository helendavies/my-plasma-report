\documentclass[11pt, oneside]{article}   	% use "amsart" instead of "article" for AMSLaTeX format
\usepackage[margin=0.75in]{geometry}                		% See geometry.pdf to learn the layout options. There are lots.
\geometry{letterpaper}                   		% ... or a4paper or a5paper or ... 
%\geometry{landscape}                		% Activate for rotated page geometry
\usepackage[parfill]{parskip}    		% Activate to begin paragraphs with an empty line rather than an indent
\usepackage{graphicx}				% Use pdf, png, jpg, or eps§ with pdflatex; use eps in DVI mode
								% TeX will automatically convert eps --> pdf in pdflatex		
\usepackage{caption}
\usepackage{subcaption}
\graphicspath{ {Users/hld523/Reports/my-plasma-report/Figures/} }								% TeX will automatically convert eps --> pdf in pdflatex		


\usepackage{amssymb}
\usepackage{amsmath}

\usepackage{color}


%SetFonts

%SetFonts


\title{Characterising the Composition of Low Temperature Air Plasma to Assess Applications for Wound Infection and Healing}
%\author{The Author}
\date{}							% Activate to display a given date or no date

\begin{document}
\maketitle
\section*{Background}
\subsection*{Wounds}

Wounds and their management are an issue both in the UK and globally \cite{Posnett2008burden}.
Methods of accelerating wound healing to decrease infection risk are sought, especially in developing countries where wound infection rates are extremely high \cite{Kihla2014risk}. 
Common wound-infecting pathogens such as \textit{Staphylococcus aureus} and \textit{pseudomonas aeruginosa} \cite{Church2006burn, Bowler2001wound} are becoming increasingly resistant to antibiotics, therefore, new wound treatments appropriate for the healthcare budgets of both developed and developing countries are required \cite{Chambers2009waves, Godebo2013multidrug, Howell2005a}.
One technique under investigation is low temperature plasma (LTP), which has shown promise for both bacterial killing and wound healing promotion \cite{Kong2009plasma, Kramer2013suitability, Isbary2012successful, Isbary2010a}.


\subsection*{Low Temperature Plasma}

Plasma is quasineutral ionised gas, considered to be the fourth state of matter \cite{Fridman2013plasmamedicine}.
Low temperature plasma (LTP), specifically, is formed when only a small percentage of gas particles are ionised, resulting in a low electron density and an overall low plasma temperature (approximately room temperature).
The appealing properties of LTP arise due to electron-mediated processes.
Firstly, dissociation of molecules by electron impact produce, for example, free radicals, including reactive oxygen and nitrogen species (RONS), which are known to be bactericidal \cite{Kong2009plasma}.
Secondly, electronic excitation from electron impact causes excitation of molecules and subsequent emission of, for example, UV radiation which is known to be bactericidal at certain wavelengths and powers \cite{Laroussi2004evaluation}.
Thirdly, sufficiently energetic free electrons can cause ionisation of other atoms/molecules to sustain the plasma and produce ions that can contribute to the bactericidal effects of LTP \cite{Mendis2000a, Laroussi2002nonthermal}.
LTP has shown significant promise in the eradication and/or inactivation of many microorganisms, including those in biofilms \cite{Laroussi2005low}. 
There is also evidence to show that LTP may promote the wound healing process in eukaryotic cells \cite{Haertel2014nonthermal, Kramer2013suitability}.
Identification of specific roles of different plasma components could help with tailoring plasmas for different biomedical applications.

Here, we propose the development of an LTP device utilising an air feed gas, which will be characterised both experimentally and computationally, and its potential for use in wound infection and healing applications assessed through various \textit{in vitro} biological assays.
The investigation into using an air feed gas is beneficial when considering low-cost wound treatments, as it should reduce running costs, and increase device portability, through eliminating the need for bottled gases.

%\textcolor{red}{Here we propose the development and characterisation of an LTP device to determine its potential for use in wound infection and healing applications. 
%The feed gas will be air to reduce the running costs and increase portability of the device, thus increasing the potential for its use in areas where low cost treatments are vital.
%This will be done through experimental and simulated characterisation of the plasma, to correlate the plasma properties with effects seen in biological assays relating to wound infection and healing.
%The computational aspects will be of use in order to compare simulated and experimental data, and to help inform the optimal experimental plasma parameters.}

%
%Here we propose the investigation of LTP for the treatment of bacterially infected wounds, in terms of its ability to eliminate bacteria, and promote the wound healing process.
%In particular, an LTP device that uses atmospheric air as its feed gas will be developed.
%By using air, expensive bottled gas is not required, therefore costs are reduced and transportability is increased. 
%This increases potential for use in areas where low-cost treatments are vital.
%\textcolor{red}{To investigate the plasma, it is proposed that an experimental characterisation of the plasma will be carried out, in terms of its RONS concentrations, emission and temperature. 
%It is hoped that, through collaboration with others, the characterisation can be extended using a global model and a pathways analysis tool \cite{Stafford2004O2, Markosyan2014PumpKin}, in order to compare simulated and experimental data, and to help inform the optimal experimental plasma parameters.}



\section*{Aims}
\begin{enumerate}
\item Develop a stable low temperature air plasma (LTAP) device
\item Characterise the plasma composition experimentally
%Experimental characterisation of plasma composition %RONS concentrations, optical emission and temperature
\item Characterise the plasma composition computationally
%Computational characterisation of plasma composition %Global plasma model and pathways analysis
\item Determine whether plasma treatment of wounded skin can accelerate wound healing%to reduce infection risk
\item Determine the tolerance to oxidative stress in prokaryotic \textit{Staphylococcus aureus} compared to eukaryotic keratinocytes
\end{enumerate}

\section*{Methods}
\subsection*{Aim 1 - Air Plasma}

Developing low temperature air plasma (LTAP) is less straightforward than noble gas plasmas (such as helium or argon) because air has a higher collisionality and shorter mean free path, thus decreasing the energy and efficiency of the system.
%This increases the number of potential collisions, thus decreasing the energy of the system and making it less efficient.
LTAP is also hotter than helium/argon plasmas, as air has a lower thermal conductivity.
To control the degree of ionisation and temperature of LTAP, it can be pulsed.
The power supply to be used produces a nano- to microsecond scale pulsed plasma, at variable, kilohertz-range frequencies, with applied voltages of approximately 20 kV. 
This means that the rise time of the waveform controlling the plasma is very fast (nanosecond scale) allowing for a well-controlled, reproducible plasma.

%The configuration of the plasma device also has an impact on how it functions.
LTAP has been shown to function in a number of different configurations.
For example, LTAP jets - so-called due to having an air feed gas which carries a 'jet' of plasma effluent, with or without charged particles, to the biological sample to be treated \cite{Kolb2008cold, Chen2009blood}.
Also, configurations where there is no air flow, but the electrodes (where the ground electrode may be the biological sample to be treated \cite{Fridman2007floating}) ignite the ambient air between them, have been presented \cite{Laroussi2004evaluation}. 
%A variation of this is shown by Fridman \textit{et al}, where the biological sample to be treated acts as the grounded electrode \cite{Fridman2007floating}.
For this project, an LTAP jet will be developed.
%with an air feed gas to carry the plasma effluent to the biological sample will be used.

%\subsubsection*{Contingency plans}
%
%If a stable air plasma cannot be produced in a reasonable amount of time, then it would be possible to change to using an argon plasma, supplemented with other gases, such as oxygen and/or water vapour.
%Argon is cheaper than helium, therefore, the aim to reduce LTP running costs would still be met.

%Due to the nature of this proposed research project, the air plasma device to be designed will utilise a flow of air through the device to carry the plasma effluent towards the sample to be treated.
%
%Whether charged particles are present in the effluent can be determined by having the air flow either perpendicular (no charged particles abs) or parallel (charged The air flow can either be parallel or perpendicular to the electrodes to 
%In particular, it might be best to have the air flow perpendicular to the electric field so that charged particles will not be present in the effluent (as these are hard to measure) and the electric field will not influence the results. ****ELectric fields and charged particles may be important but RONS are thought to be the main component causing biological effects and they are measurable so probably best to stick to them****
%This should result in biological effects that are attributable to measurable components of the plasma effluent.
\section*{Plasma Characterisation}

Through LTAP characterisation, it is hoped that the composition can be correlated with biological effects seen in the cellular assays outlined below.
%By characterising the plasma, it is hoped that the composition can be correlated with the biological effects seen in the cellular assays outlined below.
Alterations in composition can then be achieved by altering the plasma power input or the gas feed composition (for example, humid versus dry air) to see how this impacts the biological outcome, with the aim of refining the plasma for specific applications. The composition may need to differ depending on whether the aim is promote healing or eradicate bacteria. 


\subsection*{Aim 2 - Experimental Plasma Diagnostics}

Due to the high collisionality of LTAP, absorption spectroscopy techniques, including UV absorption spectroscopy and fourier transform infrared (FTIR) spectroscopy are particularly useful for quantifying RONS concentrations as they are unaffected by processes such as collisional quenching \cite{Niemi2013absolute, Schroter2015atomic}.
Laser induced fluorescence (LIF) and two-photon absorption LIF (TALIF) may also be useful as York Plasma Institute is unique in having a picosecond resolution (TA)LIF laser system, allowing for measurements of short lifetime species in high collisionality environments.
% without being too affected by collisional quenching.
Highly collisional air will still be challenging, however, through modifications to the current system, it should theoretically be possible.
Using these methods and subsequent computational analysis, absolute densities of plasma species can be determined.
Important species include ozone (O$_3$),  hydroxyl radical ($\cdot$OH), atomic oxygen (O), singlet oxygen (\textsuperscript{1}O$_2$), superoxide (O$_2$-), hydrogen peroxide (H$_2$O$_2$) and atomic nitrogen (N) \cite{Graves2014low}.

The plasma temperature can be determined either using a thermocouple, or, if charged species are present, the second positive system of nitrogen optical emission spectroscopy can be used to determine the rotational temperature of nitrogen which is almost equal to the gas temperature \cite{Twomey2011correlation}.

%UV emission by the plasma would also need to be characterised to determine whether it may be contributing to any biological effects seen.
%Treating samples with only UV and/or by placing a UV-permeable window between the plasma jet and the sample to allow only UV radiation to reach the sample (while RONS and other particles are blocked), are methods of doing this.

%The plasma temperature can be determined using a thermocouple, unless charged particles are present, in which case the second positive system of nitrogen optical emission spectroscopy can be used.
%This gives the rotational temperature of nitrogen, which is almost equal to the gas temperature \cite{Twomey2011correlation}.
%Absolute densities of these, and additional species, will be investigated both experimentally, and using a 0-dimensional chemical kinetics model.
%This model is based on a set of chemical reactions and their corresponding reaction rate coefficients. 
%It can also be used to identify the important formation and decay mechanisms for the species of interest.


\subsection*{Aim 3 - Global Plasma Model and Pathways Analysis}
For plasmas running on He/H$_2$O mixtures, a benchmarked model, using GlobalKin \cite{Stafford2004O2}, a 0-dimensional chemical kinetics model based on a set of chemical reactions and their corresponding rate coefficients, is used for comparing experimental to simulated species densities. It takes the plasma power, volume and gas flow from the experimental setup as model parameters.
%This model takes the plasma power, volume and gas flow from the experimental setup as model parameters and, as such, provides a method for comparing experimentally determined species densities to simulated densities.
Further to this, PumpKin, a tool for determining the principal reactions for the formation and destruction of species can be used \cite{Markosyan2014PumpKin}.
%Together, these tools can help inform the optimum plasma parameters to be used in the experimental setup.
However, significant work is required to make this model relevant for air plasmas, therefore, through collaboration with others, it is hoped that an air plasma model can be developed for data comparison.
%However, for this model to be applicable to air plasmas, significant work must be done to it. Therefore, through collaboration with others, it is hoped equivalent models for air plasmas can be developed for the purpose of comparison of experimental and simulated data.
Computational tools such as these are beneficial as they help with understanding the dynamics of plasma effluents, particularly with respect to reactions affecting the formation and decay of species. 
In an iterative fashion, these simulated data can then help inform optimum plasma parameters for the increase/decrease of particular, biologically relevant RONS.

%GlobalKin, to give a further understanding of experimentally measured species, and others**REF**. An additional tool, PumpKin, can then be used to identify the principal reactions causing production/destruction of the species of interest**REF**. These tools take plasma power, volume and  gas flow from the experimental setup as model parameters and as such, provide a method for comparing simulated and experimental data.
%Computational tools such as these will be beneficial to the project as they will help with the understanding of the dynamics of the plasma effluent, particularly with respect to reactions affecting the formation and decay of species. In an iterative fashion, these simulated data can then help inform optimum plasma parameters for the increase/decrease of particular, biologically relevant RONS.}


\section*{Assessing LTAP for wound healing applications}
%\section*{Assessing the Biological Effects of Air Plasma}



%A fundamental issue regarding wound management is infection and there are two main ways of tackling it, namely prevention and treatment.
%In order to prevent infection, wounds are required to heal quickly in order to rebuild the barrier to infection provided by intact skin as rapidly as possible.
%Treatment of infection is often done using antibiotics, however, overuse of antibiotics is leading to increasing antibiotic resistance, therefore alternative methods are required.
%It is possible that LTP is capable of promoting the wound healing process, as well as eradicating bacteria, therefore investigations into both are proposed.




\subsection*{Aim 4 - Promotion of Wound Healing}
To prevent infection, rapid wound healing is required.
To investigate whether LTAP can increase the speed of healing, a commercially available \textit{in vitro}, artificial skin model, EpiDermFT\texttrademark will be used (MatTek Corporation \cite{MattekWebsite}). 
Following wounding, the dynamics of the wound healing response will be measured using quantitative polymerase chain reaction (qPCR) of relevant markers.
For example, the type of keratin (K) expression by keratinocytes (a switch from K1/K10 to K6/K16/K17 on activation \cite{Pastar2014epithelialization}), and the expression of transcription factors such as Slug, which facilitates keratinocyte migration into the wound \cite{Savagner2005}, can be monitored.
The aim is to see if healing initiation occurs faster in plasma-treated versus non plasma-treated wounds. 


%The skin will be wounded and the healing response measured by assessing keratinocyte activation/gene expression over time using quantitative polymerase chain reaction (qPCR).
%%Following wounding, the first response by keratinocytes is to become activated to initiate the re-epithelialisation process. 
%Markers of interest include the type of keratin (K) expression by keratinocytes (a switch from K1/K10 to K6/K16/K17 on activation \cite{Pastar2014epithelialization}), and the expression of transcription factors such as Slug, which facilitates keratinocyte migration into the wound \cite{Savagner2005}.
%By using gene expression profiles such as these, we aim to analyse the dynamics of the wound healing response in plasma-treated and non plasma-treated wounds, to see if plasma-treated woulds initiate healing faster. 

\subsection*{Aim 5 - Plasma Cytotoxicity and Induction of Oxidative Stress}

When considering the use of LTAP in wound infection as an alternative to antibiotics (overuse of which contributes to resistance development), its effect on both bacterial and host skin cells needs to be determined. In particular, their response to harmful, plasma-induced oxidative stress (OS) \cite{Valko2007free}.
Here, we will investigate the difference in tolerance of plasma-induced OS between \textit{Staphylococcus aureus} and prokaryotic skin keratinocyte monolayers, to see if they have different OS thresholds before dying.
To assess this, cell death and markers of OS will be measured using fluorescent dyes/probes.
Cell membrane integrity, DNA damage (via detection of the DNA oxidation product 8-OHdG \cite{Valavanidis20098, Dizdaroglu2012oxidatively}) and lipid peroxidation (via detection of lipid peroxide breakdown products 4-hydroxynonenal (4-HNE) and malondialdehyde (MDA) \cite{Ayala2014lipid, Joshi2010control, Joshi2011nonthermal}) will be monitored over time following plasma treatment in both cell types.
It is important to check that a treatment with LTAP sufficient to eradicate bacteria is not going to be too harmful to the host skin cells, therefore aggravating the overall problem.


%Wound infections are often treated with antibiotics. 
%However, overuse of antibiotics contributes to increasing antibiotic resistance, therefore alternative methods, such as LTP, are required.
%The potential use of LTP on infected wounds will inevitably directly effect both the bacterial cells and host skin cells, therefore, it is important to establish the effects of LTP on bacteria, that we wish to eliminate, and skin cells, that we want to preserve, so that the overall effect is beneficial.}
%Therefore, plasma-induced oxidative stress in both eukaryotic \textit{Staphylococcus aureus} and prokaryotic skin keratinocyte monolayers will be measured as oxidative stress is known to have harmful effects \cite{Valko2007free}.
%\textcolor{blue}{To assess this, markers of oxidative stress will be measured using fluorescent dyes/probes.
%These include cell death, cell membrane integrity, DNA damage (through detection of the common DNA oxidation product 8-OHdG \cite{Valavanidis20098, Dizdaroglu2012oxidatively}) and lipid peroxidation (through detection of lipid peroxide breakdown products such as 4-hydroxynonenal (4-HNE) and malondialdehyde (MDA) \cite{Ayala2014lipid, Joshi2010control, Joshi2011nonthermal}).}
%The rationale is to correlate oxidative stress with cell death in plasma treated cells over time, and to see if bacterial and eukaryotic cells have different levels of tolerance to oxidative stress before dying.
%This is important in order to check that a treatment with LTP sufficient to eradicate bacteria is not going to be too harmful to the host skin cells, therefore aggravating the overall problem.

%%When considering the treatment of infected wounds using LTP, it is necessary to consider the effects on both eukaryotic cells (infecting bacteria) and prokaryotic cells (skin cells) as both will be implicated during the treatment process.
%To do this, three assays will be performed on plasma-treated and non plasma-treated control cells over time.
%%To assess the effects of plasma treatment on bacterial (\textit{Staphylococcus aureus}) and eukaryotic cells (keratinocytes), three assays will be performed.
%%Post plasma treatment, cells can be harvested and split into three groups and assessed simultaneously at multiple time points.
%Firstly, a fluorescent exclusion dye would be used to assess membrane integrity and cell death \cite{Kepp2011cell}.
%The remaining groups will be used to assess markers of oxidative stress (caused by RONS attack) using fluorescent probes. 
%In particular, DNA damage will be assessed via detection of the common DNA oxidation product, 8-OHdG \cite{Valavanidis20098, Dizdaroglu2012oxidatively}, and lipid peroxidation will be examined by detection of lipid peroxide breakdown products such as 4-hydroxynonenal (4-HNE) and malondialdehyde (MDA) \cite{Ayala2014lipid, Joshi2010control, Joshi2011nonthermal}.
%The rationale is to correlate oxidative stress with cell death in plasma treated cells over time, and to see if bacterial and eukaryotic cells have different levels of tolerance to oxidative stress before dying.
%This is important in order to check that a treatment with LTP sufficient to eradicate bacteria is not going to be too harmful to the host skin cells, therefore aggravating the overall problem.
%\textcolor{blue}{For wound healing applications, it is necessary to understand the effects that the plasma will have on both bacterial cells, which we wish to eliminate to sterilise the wound, and eukaryotic cells, which we want to preserve and hopefully, promote their wound healing response. }

%\textcolor{red}{
%\subsection*{Correlation of plasma and biology}
%For all the biological assays for investigating plasma effects, the plasma species concentrations can be correlated with the biological outcomes. 
%These can then be altered by, for example, altering the plasma power input through pulse width and voltage amplitude variation, or by altering the gas feed composition.
%For air plasmas, the composition of the feed gas is pretty much fixed by the composition of atmospheric air. However, whether or not water vapour is present in the feed gas can be varied and therefore, the presence of H$_2$O products can be controlled in this way. 
%Hopefully through these alterations, a set of optimal plasma parameters could be determined for specific applications. For example, the plasma parameters required to promote healing may be different to those required for bacterial elimination.}



\section*{Contingency Plans}
If a stable LTAP cannot be produced in a reasonable amount of time, then it would be possible to change to using an argon plasma, supplemented with other gases, such as oxygen and/or water vapour.
Argon is cheaper than helium, therefore, the aim to reduce LTP running costs would still be met.


\section*{Conclusions}
The novelty of this project lies in its investigation of LTAP for biomedical applications, something which has not been extensively studied previously, particularly in terms of careful plasma characterisation.
Long term, knowledge from this project will hopefully contribute to the development of a new, low-cost, portable therapy for wound infection and healing, suitable for use in the developed and developing world.



%The use of a computational model will help with understanding dynamics of the species in the plasma effluent in terms of their production/destruction, something that is hard to measure experimentally. It will also help inform how plasma parameters can be altered to influence concentrations of particular species for fine tuning of the plasma.



\bibliographystyle{ieeetr}
%\bibliographystyle{unsrt}
\bibliography{MyPapers}

\end{document}  