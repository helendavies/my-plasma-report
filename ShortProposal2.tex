\documentclass[11pt, oneside]{article}   	% use "amsart" instead of "article" for AMSLaTeX format
\usepackage[margin=0.75in]{geometry}                		% See geometry.pdf to learn the layout options. There are lots.
\geometry{letterpaper}                   		% ... or a4paper or a5paper or ... 
%\geometry{landscape}                		% Activate for rotated page geometry
\usepackage[parfill]{parskip}    		% Activate to begin paragraphs with an empty line rather than an indent
\usepackage{graphicx}				% Use pdf, png, jpg, or eps§ with pdflatex; use eps in DVI mode
								% TeX will automatically convert eps --> pdf in pdflatex		
\usepackage{caption}
\usepackage{subcaption}
\graphicspath{ {Users/hld523/Reports/my-plasma-report/Figures/} }								% TeX will automatically convert eps --> pdf in pdflatex		


\usepackage{amssymb}
\usepackage{amsmath}

\usepackage{color}


%SetFonts

%SetFonts


\title{PhD Proposal}
%\author{The Author}
\date{}							% Activate to display a given date or no date

\begin{document}
\maketitle
\section*{Background}
\subsection*{Wounds}

Wounds and their management are an issue both in the UK and globally \cite{Posnett2008burden}.
Methods of accelerating wound healing to decrease infection risk are sought, especially in developing countries where wound infection rates are extremely high \cite{Kihla2014risk}. 
Common wound-infecting pathogens such as \textit{Staphylococcus aureus} and \textit{pseudomonas aeruginosa} \cite{Church2006burn, Bowler2001wound} are becoming increasingly resistant to antibiotics, therefore, new wound treatments appropriate for the healthcare budgets of both developed and developing countries are required \cite{Chambers2009waves, Godebo2013multidrug, Howell2005a}.
One technique under investigation is low temperature plasma (LTP), which has shown promise for both bacterial killing and wound healing promotion \cite{Kong2009plasma, Kramer2013suitability, Isbary2012successful, Isbary2010a}.


\subsection*{Low Temperature Plasma}

Plasma is quasineutral ionised gas, considered to be the fourth state of matter \cite{Fridman2013plasmamedicine}.
Low temperature plasma (LTP), specifically, is formed when only a small percentage of gas particles are ionised, resulting in a low electron density and an overall low plasma temperature (approximately room temperature).
The appealing properties of LTP arise due to electron-mediated processes.
Firstly, dissociation of molecules by electron impact produce, for example, free radicals, including reactive oxygen and nitrogen species (RONS), which are known to be bactericidal \cite{Kong2009plasma}.
Secondly, electronic excitation from electron impact causes excitation of molecules and subsequent emission of, for example, UV radiation which is known to be bactericidal at certain wavelengths and powers \cite{Laroussi2004evaluation}.
Thirdly, sufficiently energetic free electrons can cause ionisation of other atoms/molecules to sustain the plasma and produce ions that can contribute to the bactericidal effects of LTP \cite{Mendis2000a, Laroussi2002nonthermal}.
LTP has shown significant promise in the eradication and/or inactivation of many microorganisms, including those in biofilms \cite{Laroussi2005low}. 
There is also evidence to show that LTP may promote the wound healing process in eukaryotic cells \cite{Haertel2014nonthermal, Kramer2013suitability}.
Identification of specific roles of different plasma components could help with tailoring plasmas for different biomedical applications.

Here we propose the investigation of LTP for the treatment of bacterially infected wounds, in terms of its ability to eliminate bacteria, and promote the wound healing process.
In particular, an LTP device that uses atmospheric air as its feed gas will be developed.
By using air, expensive bottled gas is not required, therefore costs are reduced and transportability is increased. 
This increases potential for use in areas where low-cost treatments are vital.
\textcolor{red}{To investigate the plasma, it is proposed that an experimental characterisation of the plasma will be carried out, in terms of its RONS concentrations, emission and temperature. 
This characterisation can be extended using a global model and a pathways analysis tool \cite{Stafford2004O2, Markosyan2014PumpKin}, in order to compare simulated and experimental data, and to help inform the optimal experimental plasma parameters.}



\section*{Aims}
\begin{enumerate}
\item Develop a stable low temperature air plasma device
\item Experimental characterisation of plasma RONS concentrations, optical emission and temperature
\item Global plasma model and pathways analysis
\item Determine the tolerance to oxidative stress in prokaryotic \textit{Staphylococcus aureus} compared to eukaryotic keratinocytes
\item Determine whether plasma treatment of wounded skin can accelerate the initial wound healing response
\end{enumerate}

\section*{Methods}
\subsection*{Aim 1 - Air Plasma}

Developing LTP that uses air as a feed gas is less straightforward than using noble gases, such as helium or argon, because air has a higher collisionality and shorter mean free path.
This increases the number of potential collisions, thus decreasing the energy of the system and making it less efficient.
Air plasmas are also hotter than helium/argon plasmas, as air has a lower thermal conductivity.
To control the degree of ionisation and temperature of air plasma, it can be pulsed.
The power supply to be used produces a nano- to microsecond scale pulsed plasma, at variable, kilohertz-range frequencies, with applied voltages of approximately 20 kV. 
This means that the rise time of the waveform controlling the plasma is very fast (nanosecond scale) allowing for a well-controlled, reproducible plasma.

%The configuration of the plasma device also has an impact on how it functions.
Air plasmas have been shown to function in a number of different configurations.
For example, air plasma jets - so-called due to having an air feed gas which carries a 'jet' of plasma effluent, with or without charged particles, to the biological sample to be treated \cite{Kolb2008cold, Chen2009blood}.
On the other hand, configurations where there is no air flow, but the electrodes ignite the ambient air between them have also been presented \cite{Laroussi2004evaluation}. 
A variation of this is shown by Fridman \textit{et al}, where the biological sample to be treated acts as the grounded electrode \cite{Fridman2007floating}.
For this project, a plasma jet with an air feed gas to carry the plasma effluent to the biological sample will be used.

%Due to the nature of this proposed research project, the air plasma device to be designed will utilise a flow of air through the device to carry the plasma effluent towards the sample to be treated.
%
%Whether charged particles are present in the effluent can be determined by having the air flow either perpendicular (no charged particles abs) or parallel (charged The air flow can either be parallel or perpendicular to the electrodes to 
%In particular, it might be best to have the air flow perpendicular to the electric field so that charged particles will not be present in the effluent (as these are hard to measure) and the electric field will not influence the results. ****ELectric fields and charged particles may be important but RONS are thought to be the main component causing biological effects and they are measurable so probably best to stick to them****
%This should result in biological effects that are attributable to measurable components of the plasma effluent.

\subsection*{Aim 2 - Plasma Diagnostics}

Due to the high collisionality of air plasmas, absorption spectroscopy techniques, including UV absorption spectroscopy and fourier transform infrared (FTIR) spectroscopy are particularly useful for quantifying RONS concentrations as they are unaffected by processes such as collisional quenching \cite{Niemi2013absolute, Schroter2015atomic}.
However, laser induced fluorescence (LIF) and two-photon absorption LIF (TALIF) may also be useful as York Plasma Institute is unique in having a picosecond resolution (TA)LIF laser system, allowing for measurements of short lifetime species in high collisionality environments.
% without being too affected by collisional quenching.
Highly collisional air will still be challenging, however, through modifications to the current system, it should theoretically be possible.
\textcolor{red}{Using methods such as these, and subsequent computational methods to analyse the raw data, absolute densities of plasma species can be determined.}
Important species to measure include ozone (O$_3$),  hydroxyl radical ($\cdot$OH), atomic oxygen (O), singlet oxygen (\textsuperscript{1}O$_2$), superoxide (O$_2$-), hydrogen peroxide (H$_2$O$_2$),  atomic nitrogen (N), nitric oxide (NO), peryoxynitrite (ONOO-), and other NO$_x$ species \cite{Graves2014low}. 

UV emission by the plasma would also need to be characterised to determine whether it may be contributing to any biological effects seen.
Treating samples with only UV and/or by placing a UV-permeable window between the plasma jet and the sample to allow only UV radiation to reach the sample (while RONS and other particles are blocked), are methods of doing this.

The plasma temperature can be determined using a thermocouple, unless charged particles are present, in which case the second positive system of nitrogen optical emission spectroscopy can be used.
This gives the rotational temperature of nitrogen, which is almost equal to the gas temperature \cite{Twomey2011correlation}.
%Absolute densities of these, and additional species, will be investigated both experimentally, and using a 0-dimensional chemical kinetics model.
%This model is based on a set of chemical reactions and their corresponding reaction rate coefficients. 
%It can also be used to identify the important formation and decay mechanisms for the species of interest.

\textcolor{red}{
\subsection*{Aim 3 - Computational Modelling}
The plasma characterisation can be made more thorough by using a global model, such as GlobalKin \cite{Stafford2004O2}, a 0-dimensional chemical kinetics model based on a set of chemical reactions and their corresponding rate coefficients.
This model takes the plasma power, volume and gas flow from the experimental setup as model parameters and, as such, provides a method for comparing experimentally determined species densities to simulated densities.
Further to this, PumpKin, a tool for determining the principal reactions for the formation and destruction of species can be used \cite{Markosyan2014PumpKin}.
Together, these tools can help inform the optimum plasma parameters to be used in the experimental setup.
Computational tools such as these will be beneficial to the project as they will help with the understanding of the dynamics of the plasma effluent, particularly with respect to reactions affecting the formation and decay of species. In an iterative fashion, these simulated data can then help inform optimum plasma parameters for the increase/decrease of particular, biologically relevant RONS.}

%GlobalKin, to give a further understanding of experimentally measured species, and others**REF**. An additional tool, PumpKin, can then be used to identify the principal reactions causing production/destruction of the species of interest**REF**. These tools take plasma power, volume and  gas flow from the experimental setup as model parameters and as such, provide a method for comparing simulated and experimental data.
%Computational tools such as these will be beneficial to the project as they will help with the understanding of the dynamics of the plasma effluent, particularly with respect to reactions affecting the formation and decay of species. In an iterative fashion, these simulated data can then help inform optimum plasma parameters for the increase/decrease of particular, biologically relevant RONS.}



%\section*{Assessing the Biological Effects of Air Plasma}

\subsection*{Aim 4 - Plasma Cytotoxicity and Induction of Oxidative Stress}
To assess the effects of plasma treatment on bacterial (\textit{Staphylococcus aureus}) and eukaryotic cells (keratinocytes), three assays will be performed.
Post plasma treatment, cells can be harvested and split into three groups and assessed simultaneously at multiple time points.
Firstly, a fluorescent exclusion dye would be used to assess membrane integrity and cell death \cite{Kepp2011cell}.
The remaining groups will be used to assess markers of oxidative stress (caused by RONS attack) using fluorescent probes. 
In particular, DNA damage will be assessed via detection of the common DNA oxidation product, 8-OHdG \cite{Valavanidis20098, Dizdaroglu2012oxidatively}, and lipid peroxidation will be examined by detection of lipid peroxide breakdown products such as 4-hydroxynonenal (4-HNE) and malondialdehyde (MDA) \cite{Ayala2014lipid, Joshi2010control, Joshi2011nonthermal}.
The rationale is to correlate oxidative stress with cell death in plasma treated cells over time, and to see if bacterial and eukaryotic cells have different levels of tolerance to oxidative stress before dying.



\subsection*{Aim 5 - Promotion of Wound Healing}
%Overuse of antibiotics is contributing to the increasing problem of antibiotic resistance, therefore, if wound healing can be accelerated, infection rates should decrease, therefore reducing antibiotic requirement.
%Here, we aim to investigate the effects of LTP on the speed of wound healing, in particular the initiation of the healing process.
To investigate whether LTP can increase the speed of healing, a commercially available \textit{in vitro}, artificial full-thickness skin model, EpiDermFT\texttrademark will be used (MatTek Corporation \cite{MattekWebsite}). 
The skin will be wounded and the healing response measured by assessing gene expression over time using quantitative polymerase chain reaction (qPCR).
Following wounding, the first response by keratinocytes is to become activated to initiate the re-epithelialisation process. 
Markers of interest include the type of keratin (K) expression by keratinocytes (a switch from K1/K10 to K6, K16 and K17 on activation \cite{Pastar2014epithelialization}), and the expression of transcription factors such as Slug, which facilitates keratinocyte migration into the wound \cite{Savagner2005}.
By using gene expression profiles such as these, we aim to analyse the dynamics of the wound healing response in plasma-treated and non plasma-treated wounds, to see if plasma-treated woulds initiate healing faster.
If wounds can be influenced to heal faster, the risk of infection and requirement for antibiotic use should reduce.


\section*{Contingency Plans}

If a stable air plasma cannot be produced in a reasonable amount of time, then it would be possible to change to using an argon plasma, supplemented with other gases, such as oxygen and/or water vapour.
Argon is cheaper than helium, therefore, the aim to reduce LTP running costs would still be met.

\section*{Conclusions}
This novelty of this project lies in its investigation of air plasmas for biomedical applications, something which has not been extensively studied previously, particularly in terms of careful plasma characterisation.
Following design and characterisation of an air plasma, it is hoped that plasma properties and their biological effects on both eukaryotic and prokaryotic cells can be correlated. 

For wound healing applications, it is necessary to understand the effects that the plasma will have on both bacterial cells, which we wish to eliminate to sterilise the wound, and eukaryotic cells, which we want to preserve and hopefully, promote their wound healing response. 
The use of a computational model will help with understanding dynamics of the species in the plasma effluent in terms of their production/destruction, something that is hard to measure experimentally. It will also help inform how plasma parameters can be altered to influence concentrations of particular species for fine tuning of the plasma.
Long term, knowledge from this project will hopefully contribute to the development of a new, low-cost, portable wound treatment technique, suitable for use in the developed and developing world.



\bibliographystyle{ieeetr}
%\bibliographystyle{unsrt}
\bibliography{MyPapers}

\end{document}  