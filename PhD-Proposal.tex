\documentclass[11pt, oneside]{article}   	% use "amsart" instead of "article" for AMSLaTeX format
\usepackage[margin=0.75in]{geometry}                		% See geometry.pdf to learn the layout options. There are lots.
\geometry{letterpaper}                   		% ... or a4paper or a5paper or ... 
%\geometry{landscape}                		% Activate for rotated page geometry
\usepackage[parfill]{parskip}    		% Activate to begin paragraphs with an empty line rather than an indent
\usepackage{graphicx}				% Use pdf, png, jpg, or eps§ with pdflatex; use eps in DVI mode
								% TeX will automatically convert eps --> pdf in pdflatex		
\usepackage{caption}
\usepackage{subcaption}
\graphicspath{ {Users/hld523/Reports/my-plasma-report/Figures/} }								% TeX will automatically convert eps --> pdf in pdflatex		


\usepackage{amssymb}
\usepackage{amsmath}

%SetFonts

%SetFonts


\title{PhD Proposal}
%\author{The Author}
\date{}							% Activate to display a given date or no date

\begin{document}
\maketitle
\section*{Background}
\subsection*{Wounds}

In the UK, wounds and their management place a huge burden on the national health system \cite{Posnett2008burden}. Further to this, patients with wounds can experience significant decreases in psychological and physical well-being, particularly if their wounds become chronic \cite{Franks2003health}. 
Infections by common pathogens such as \textit{Staphylococcus aureus, Pseudomonas aeruginosa} and beta-haemolytic streptococci \cite{Bowler2001wound} within wounds can further complicate the healing process. Since pathogens such as these are becoming increasingly resistant to antibiotics**REF**, it is of paramount importance to be able to identify new modalities capable of helping with the treatment of persistent, bacterially infected wounds.
Once such modality may be the use of low temperature atmospheric pressure plasma (LTP), which has shown promise in the past for both bacterial inactivation and wound healing promotion \cite{Kong2009plasma, Kramer2013suitability, Isbary2012successful, Isbary2010a}.

%The normal wound healing process follows the progression of (1) coagulation and haemostasis, (2) inflammation, (3) proliferation and (4) wound remodelling \cite{Velnar2009the}. However, sometimes this process can be interrupted and result in a wound that takes longer to heal. This can also be further complicated by colonisation of the wound by bacteria and other pathogens.  Common bacteria responsible for wound infections include \textit{Staphylococcus aureus, Pseudomonas aeruginosa} and beta-haemolytic streptococci \cite{Bowler2001wound}. Increasingly, antibiotic resistance of these pathogens tests our wound management protocols, reducing the availability of antimicrobial substances to clear the wound bed to allow healing to continue. It is therefore of paramount importance to be able to identify new modalities capable of helping with the treatment of persistent, bacterially infected wounds.
%Here, it is proposed that low temperature plasma may be an extremely useful technology for such treatment as it has shown significant promise in the past for both bacterial inactivation and healing promotion \cite{Kong2009plasma, Kramer2013suitability, Isbary2012successful, Isbary2010a}.

\subsection*{Low Temperature Plasma}

%Plasma is considered to be the fourth state of matter, formed when the energy of a gas is increased, such that some gas particles become ionised, releasing electrons. 
%The number of electrons released (degree of ionisation), determines the properties of the plasma, as electrons mediate the very important reaction

%These free electrons mediate many reactions that occur within plasma, including dissociation (splitting of molecules into, for example, free radicals), electronic excitation (where electron impact causing excitation of a particle to a higher energy level, which subsequently returns to its ground state, releasing energy in the form of a photon) and ionisation. Plasmas can exist at high temperatures, for example, the sun and lightening. 
%However, plasmas can be formed at low temperatures, and these are characterised by not being in thermodynamic equilibrium (electron temperature far hotter than the gas and, therefore, the overall system) and by having a low electron density ($\approx$ 0.1 - 1\% of gas particles ionised) \cite{Fridman2013plasmamedicine, Moreau2008nonthermal}.
%The low temperature nature of LTP makes it suitable for direct application to biological samples, such as the skin, without causing damage. Hence the potential use of LTP in biomedical applications.



Plasma is considered to be the fourth state of matter, formed when the energy of a gas is increased, such that some gas particles become ionised, releasing electrons. These free electrons mediate many reactions that occur within plasma, including dissociation (splitting of molecules into, for example, free radicals), electronic excitation (where electron impact causing excitation of a particle to a higher energy level, which subsequently returns to its ground state, releasing energy in the form of a photon) and ionisation. Plasmas can exist at high temperatures, for example, the sun and lightening. 
In plasmas such as these, the degree of ionisation (the number of gas particles that are ionised) is close to 100\% and all the particles within the plasma are in thermal equilibrium, such that $T_e \approx T_i \approx T_g$ (where $T_e$ is electron temperature, $T_i$ is ion temperature and $T_g$ is gas temperature).
However, plasmas with very low electron densities ($\approx$ 0.1 - 1\%) can also form and these plasmas never reach thermal equilibrium because the energy in the system is less. Thus $T_e >> T_i \approx T_g$, meaning that the overall temperature of the plasma remains relatively cool, around room temperature \cite{Fridman2013plasmamedicine, Moreau2008nonthermal}. These plasmas are known as low temperature plasmas (LTP). LTP is of a suitable temperature to be applied directly to biological samples, such as the skin, and opens up the potential for its use in many biomedical applications.


Low temperature plasmas are efficient producers of reactive oxygen and nitrogen species (RONS). The exact composition of RONS in the plasma depends on the composition of the feed gas. Important species present in plasmas (depending on feed gas composition) include ozone (O$_3$), hydroxyl radical ($\cdot$OH), atomic oxygen (O), singlet oxygen (\textsuperscript{1}O$_2$), superoxide (0$_2$-), hydrogen peroxide (H$_2$O$_2$),  atomic nitrogen (N), nitric oxide (NO), peryoxynitrite (ONOO-), and other NO$_x$ species \cite{Graves2012the}. 

This potent production of RONS by LTP is one of the main features which means it lends itself well to biomedical applications, in particular for bacterial inactivation/killing \cite{Kong2009plasma}. LTP has been shown to be very effective for the eradication and/or inactivation of many microorganisms, including those in biofilms \cite{Laroussi2005low}. 
%Both gram negative and gram positive bacteria seem to be susceptible to killing by LTP, however, gram negative have been shown to be the most effectively killed (Laroussi2003plasma). This is thought to be due to the difference in the membrane composition between gram positive and gram negative bacterial. 
Besides RONS, UV radiation produced by the plasma may play a role, along with charged particles and electric fields, also produced by the plasma \cite{Mendis2000a, Laroussi2002nonthermal}. The elucidation of the roles of these different plasma components would provide useful information as to how plasmas could be tailored to specific biomedical applications. Alongside this ability to kill bacteria, there is also evidence to show that plasmas may have beneficial effects on the promotion of wound healing in eukaryotic cells \cite{Haertel2014nonthermal, Kramer2013suitability} with particular respect to increasing proliferation and migration of eukaryotic cells involved in the healing process, through altering cell surface expression of associated molecules. It is believed that the observed effects of LTP are mainly due to the plasma-produced RONS.


Here we propose the development of a new air plasma device and aim to characterise its cytotoxic effects on both bacterial and eukaryotic cells, such as common wound-colonising bacteria and skin epithelial cells, respectively. Also, the effects that LTP has on the wound healing response in epithelial cells will be investigated.

\section*{Aims}
\begin{enumerate}
\item Develop a stable air plasma device with suitable configuration
\item Characterise the plasma device in terms of RONS production, UV emission and temperature
\item Determine the death rates of bacterial and eukaryotic cells treated with plasma to see if there is a difference between cell types. This will be assessed for different plasma parameters, for example plasma power, presence of water in the air feed gas etc
\item Determine the wound healing promotion response, if any, in skin epithelial cells treated with plasma
\end{enumerate}

\section*{Methods}
\subsection*{Aim 1 - Air Plasma}

Commonly, LTP use a noble gas such as helium or argon as a feed gas. However, this reliance on specific gases results in expensive plasma devices that have limited portability when considering potential applications in less well-off parts of the world. 
For this reason developing a stable plasma device that runs on a feed gas of atmospheric air is important. 
The reason that this is not as straight forward as simply swapping gas supplies, is because of the high collisionality of air compared the helium or argon, and the much shorter mean free path of electrons present in the plasma. This means that electrons can have far more interactions with particles present in air in terms of potential for dissociation, excitation and ionisation. For example, there are many more excited states and metastables that can be produced by atoms/molecules present in air compared to those in helium/argon.

Another property to consider when looking at different feed gases is the thermal conductivity of the gas. This is important as it helps to determine the overall temperature of the plasma. This is of interest when designing plasmas for biomedical applications that could be in contact with delicate biological samples, such as the skin. For example, helium has a higher thermal conductivity compared to air, meaning that the overall temperature of plasmas run on helium is lower than those run on air. 

To control the degree of ionisation and temperature of air plasmas, the energy input can be controlled using a pulsed plasma, where there are short pulses of high voltage.
For the proposed air plasma device, the existing power supply is able to produce a nano- to microsecond scale pulsed plasma, at variable frequencies in the kilohertz range with applied voltages of approximately 20 kV. This means that the rise time of the wave form controlling the plasma is very fast (nanosecond scale) which is good as it allows for a well-controlled, reproducible plasma discharge.

To produce a stable air plasma device, there are different possible configurations of electrodes, noteably, dielectric barrier discharges (DBD) and plasma jets. Laroussi \textit{et al} shows an air plasma in a DBD configuration, whereby there is a powered electrode and a grounded electrode and the ambient air in between becomes ionised to form the plasma \cite{Laroussi2004evaluation}. In a different study by Fridman \textit{et al}, an air plasma is shown using a modified DBD, whereby the grounded electrode is the biological sample such as the skin. In this case, the powered electrode is covered with a dielectric and the gas forming the plasma is ambient air \cite{Fridman2007floating}. In the cases of DBD plasmas, there is no gas flow, therefore, the treatment sample is required to be in contact with the plasma as there is no air flow to carry plasma effluent towards the sample. This contrasts with plasma jets which use an air flow between two electrodes to ignite a plasma, then the gas flow carries plasma effluent, containing reactive species, with or without charged particles depending on the electrode configuration, to the sample. Plasmas such as these, using atmospheric air as a feed gas have been presented previously by the likes of Kolb \textit{et al} and Chen \textit{et al} \cite{Kolb2008cold, Chen2009blood}.

Due to the nature of this proposed research project, it is likely that the air plasma device to be designed will be a plasma jet, utilising the flow of air through the device to carry the plasma effluent towards the bacterial sample to be treated. It is then important to decide whether the air flow will be parallel or perpendicular to the electric field as this influences whether charged particles and electric fields can also act on the biological sample. As these components are hard to measure, it may be most suitable to use an airflow perpendicular to the electrodes so that biological effects seen are more likely to be due to measurable components of the plasma, not charged particles and electric fields.

\subsection*{Aim 2 - Plasma Diagnostics}
In order to quantify the abundance of RONS, techniques such as absorption spectroscopy and laser induced fluorescence (LIF) can be used. 
Due to the high collisionality of air plasmas, absorption spectroscopy techniques, including UV absorption spectroscopy and fourier transform infrared (FTIR) spectroscopy are particularly useful as they are unaffected by processes such as collisional quenching \cite{Niemi2013absolute, Schroter2015atomic}.
LIF, and variations of, such as two-photon LIF, are more affected by collisional quenching and the lifetime of the species being measured have to be within the resolution of the LIF measurements.
York Plasma Institute is unique in that the laser system used for (TA)LIF measurements has picosecond resolution, allowing measurements of reactive species in high collisionality environments.

%In order to quantify the abundance of reactive oxygen and nitrogen species, techniques such as absorption spectroscopy, and fourier transform infrared (FTIR) spectroscopy can be used. Techniques such as these are commonly used in atmospheric pressure plasmas, over methods such as (two-photon absorption) laser induced fluorescence ((TA)LIF), as they are unaffected by the high collisionality present in atmospheric pressure plasmas \cite{Niemi2013absolute, Schroter2015atomic}.
Important species to measure include ozone (O$_3$),  hydroxyl radical ($\cdot$OH), atomic oxygen (O), singlet oxygen (\textsuperscript{1}O$_2$), superoxide (0$_2$-), hydrogen peroxide (H$_2$O$_2$),  atomic nitrogen (N), nitric oxide (NO), peryoxynitrite (ONOO-), and other NO$_x$ species \cite{Graves2014low}. 
In addition to RONS production, the UV radiation emitted by the plasma would also need to be investigated to determine whether this is also contributing to any observed biological effects of plasma treatment. This has been done before either by treating the sample with only UV radiation with a UV LED, or by placing a UV permeable window between the plasma jet and the sample to allow only UV radiation to reach the sample, while RONS and other particles are blocked.

Finally, it is vital to understand the temperature of the plasma/plasma effluent in contact with the biological sample.
One good way to measure temperature is to use a thermocouple. However, this is not so reliable when charged particles are present. If this is likely to be a problem, an alternative method would be to use the second positive system of nitrogen optical emission spectroscopy to find the rotational temperature (T$_{rot}$) of N$_2$. This is useful as the gas temperature is almost equal to the T$_{rot}$ of N$_2$ \cite{Twomey2011correlation}.


\section*{Cell Assays}

\subsection*{Aim 3 - Plasma cytotoxicity}
The first biological assay would be to determine the rates of cell death in bacterial cells compared to eukaryotic cells, to see if there is a difference.
It is proposed that both types of cells would be treated with plasma, harvested post treatment, then split into three separate groups to assess three things. Firstly, cell death rates, secondly, DNA damage (a common harmful consequence of RONS attacking cells) and finally lipid peroxidation (another harmful process mediated by RONS).

To measure cytotoxicity, the cells would be stained with a fluorescent exclusion dye. The fluorescence of the cells could then give an indication of the membrane integrity and survival rates of cells \cite{Kepp2011cell}. Ideally, this would be repeated at multiple time points (for both plasma treated and non plasma treated controls) to see if the cytotoxic effects of plasma are long or short lived.
Alongside this, at the same time points, DNA damage (via detection of 8-OHdG, a DNA oxidation product, widely used as a marker of oxidative stress \cite{Valavanidis20098, Dizdaroglu2012oxidatively}) and lipid peroxidation (via the detection of lipid peroxide breakdown products such as 4-hydroxynonenal (4-HNE) and malondialdehyde (MDA) \cite{Ayala2014lipid} using fluorescent probes) can be measured \cite{Joshi2010control, Joshi2011nonthermal}. 
The rationale behind this is to look, at the population level, to correlate the DNA death rates with the levels of DNA damage and lipid peroxidation over time. This could indicate whether bacterial and eukaryotic cells can tolerate similar levels of DNA damage and lipid peroxidation before dying. For example, if levels of DNA damage and lipid peroxidation in both types of cells are similar, but the death rate in eukaryotic cells is lower, this may suggest that eukaryotic cells have higher levels of tolerance to these harmful processes. This could potentially equate to some specificity in plasma treatment resulting in bacterial cells being more harmed than eukaryotic cells.

\subsection*{Aim 4 - Promotion of Wound Healing}
Artificial skin can be used to study the wound healing response of epithelial cells at the edge of the wound \cite{Rasmussen2013classical, Garlick2007engineering}. Following wounding, epithelial cells must change from being fully differentiation stable cells to instead become controlled proliferating, migrating cells instead. This prepares the cells for re-epithelialisation.
The idea is to determine whether LTP can influence this change to increase the speed at which epithelial cells become activated. If LTP can influence this process, it would suggest that it might be possible in the case of non-healing wounds where the healing process has paused, to reawaken the healing activity of epithelial cells.
This would be beneficial because the faster wounds heal, the lower the chance of infection which results in a decreased need for antibiotic treatment.
It would, therefore, be interesting to look at the cellular responses in skin epithelial cells (in the artificial skin model) both with and without plasma treatment in response to wounding. Garlick \textit{et al} presents some factors that signify this preparation for re-epithelialisation. 
For example, the wounding response results in upregulation of proliferative markers such as Ki-67, a decrease in the proportion of cells in S-phase and an upregulation of $\alpha$5 integrin on epithelial cells during the wound response \cite{Garlick2007engineering}. By looking at the time profile of gene expression relating to these factors, using quantitative polymerase chain reaction (qPCR), it may be possible to detect any differences in plasma treated epithelial cells compared with non treated epithelial cells.



\bibliographystyle{unsrt}
\bibliography{/Users/hld523/Bibliography/MyPapers}

\end{document}  