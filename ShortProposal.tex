\documentclass[11pt, oneside]{article}   	% use "amsart" instead of "article" for AMSLaTeX format
\usepackage[margin=0.75in]{geometry}                		% See geometry.pdf to learn the layout options. There are lots.
\geometry{letterpaper}                   		% ... or a4paper or a5paper or ... 
%\geometry{landscape}                		% Activate for rotated page geometry
\usepackage[parfill]{parskip}    		% Activate to begin paragraphs with an empty line rather than an indent
\usepackage{graphicx}				% Use pdf, png, jpg, or eps§ with pdflatex; use eps in DVI mode
								% TeX will automatically convert eps --> pdf in pdflatex		
\usepackage{caption}
\usepackage{subcaption}
\graphicspath{ {Users/hld523/Reports/my-plasma-report/Figures/} }								% TeX will automatically convert eps --> pdf in pdflatex		


\usepackage{amssymb}
\usepackage{amsmath}

%SetFonts

%SetFonts


\title{PhD Proposal}
%\author{The Author}
\date{}							% Activate to display a given date or no date

\begin{document}
\maketitle
\section*{Background}
\subsection*{Wounds}

In the UK, wounds and their management place a huge burden on the national health system \cite{Posnett2008burden}. 
Furthermore, patients with wounds can experience significant decreases in psychological and physical well-being, particularly if their wounds become infected by common pathogens such as \textit{Staphylococcus aureus, Pseudomonas aeruginosa} and beta-haemolytic streptococci \cite{Franks2003health, Bowler2001wound}. 
With increasing antibiotic resistance, new strategies for treating infected, chronic wounds are vital.
%Since pathogens such as these are becoming increasingly resistant to antibiotics**REF**, it is of paramount importance to be able to identify new modalities capable of helping with the treatment of persistent, bacterially infected wounds.
One such modality may be low temperature plasma (LTP) treatment, which has shown promise for both bacterial inactivation and wound healing promotion \cite{Kong2009plasma, Kramer2013suitability, Isbary2012successful, Isbary2010a}.

%The normal wound healing process follows the progression of (1) coagulation and haemostasis, (2) inflammation, (3) proliferation and (4) wound remodelling \cite{Velnar2009the}. However, sometimes this process can be interrupted and result in a wound that takes longer to heal. This can also be further complicated by colonisation of the wound by bacteria and other pathogens.  Common bacteria responsible for wound infections include \textit{Staphylococcus aureus, Pseudomonas aeruginosa} and beta-haemolytic streptococci \cite{Bowler2001wound}. Increasingly, antibiotic resistance of these pathogens tests our wound management protocols, reducing the availability of antimicrobial substances to clear the wound bed to allow healing to continue. It is therefore of paramount importance to be able to identify new modalities capable of helping with the treatment of persistent, bacterially infected wounds.
%Here, it is proposed that low temperature plasma may be an extremely useful technology for such treatment as it has shown significant promise in the past for both bacterial inactivation and healing promotion \cite{Kong2009plasma, Kramer2013suitability, Isbary2012successful, Isbary2010a}.

\subsection*{Low Temperature Plasma}



Plasma is quasineutral ionised gas, considered to be the fourth state of matter.
Low temperature plasma (LTP), specifically, is formed when only a small percentage of gas particles are ionised, resulting in a low electron density and an overall plasma temperature of approximately room temperature.
%and the overall temperature of the plasma is kept around room temperature as thermodynamic equilibrium cannot be reached due to the relatively small number of hot electrons.
The appealing properties of LTP arise due to electron mediated processes.
%Electrons in the plasma give LTP the properties that give it promise in biomedical applications.
Firstly, dissociation of molecules by electron impact produce, for example, free radicals. This makes LTP a effective producer of RONS, which are known to be bactericidal \cite{Kong2009plasma}.
%meaning LTP is a potent producer of reactive oxygen and nitrogen species (RONS). 
%Plasma-produced RONS are known to be bactericidal \cite{Kong2009plasma}. 
Secondly, electronic excitation from electron impact causes excitation of molecules and subsequent emission of, for example, UV radiation which is known to be bactericidal at certain wavelengths and powers \cite{Laroussi2004evaluation}.
%decay back to ground state with photon emission result in emission of, for example, UV radiation which is known to be bactericidal at certain wavelengths and powers**REF**.
Thirdly, sufficiently energetic free electrons can cause ionisation of other atoms/molecules through collisions. Ionisation sustains the plasma and produces ions that can contribute to the bactericidal effects of LTP \cite{Mendis2000a, Laroussi2002nonthermal}.
This combination of LTP properties has shown significant promise in the eradication and/or inactivation of many microorganisms, including those in biofilms \cite{Laroussi2005low}. 
There is also evidence to show that plasmas may have beneficial effects on the promotion of wound healing in eukaryotic cells \cite{Haertel2014nonthermal, Kramer2013suitability}. %with particular respect to initiating the wound healing response.
%increasing proliferation and migration of eukaryotic cells involved in the healing process, through altering cell surface expression of associated molecules.
Further investigation into the specific roles of different plasma components could help with tailoring plasmas for specific applications.
%Altogether, further investigation into LTP and the elucidation of the roles of these different plasma components would provide useful information as to how plasmas could be tailored to specific biomedical applications.

Here we propose the investigation of LTP for the treatment of bacterially infected wounds, in terms of its ability to eliminate bacteria, and promote the wound healing process.
In particular, an LTP device that uses atmospheric air as its feed gas will be developed.
Air plasmas are of particular interest as they would eliminate the need for expensive bottled gas, therefore, reducing the cost of plasma treatments and increasing the transportability of plasma devices.
This could potentially allow LTP treatments to be taken to areas of the world where low cost treatments are vital.


%it will be investigated how an LTP device can be developed that will use atmospheric air as its feed gas. 
%Investigation into air plasmas is of use as this would eliminate the need for expensive bottled gases, and increase portability of plasma devices, thus opening up the opportunity for cheaper plasma treatment options, and potentially allow LTP treatment to be taken to areas of the world where low cost treatments are vital.


\section*{Aims}
\begin{enumerate}
\item Develop a stable air plasma device
\item Characterise the RONS production, UV emission and temperature of the plasma both experimentally and using a global model
\item Determine the tolerance to oxidative stress in \textit{Staphylococcus aureus} compared to keratinocytes.
\item Determine the wound healing promotion response, if any, in skin epithelial cells treated with plasma
\end{enumerate}

\section*{Methods}
\subsection*{Aim 1 - Air Plasma}

Developing LTP that uses air as a feed gas is less straightforward than using noble gases such as helium or argon due to air having a higher collisionality and shorter electron mean free path.
%This is due to air having a higher collisionality and much shorter electron mean free path compared to, for example, helium.
This increases the number of potential collisions by electrons, thus decreasing the energy of the system and making it less efficient.
Air plasmas are also hotter than helium/argon plasmas due to air having a lower thermal conductivity.
%Another important consideration when using air plasmas for biomedical applications is the fact that air has a lower thermal conductivity than helium, meaning that the overall plasma is hotter.
To control the degree of ionisation and temperature of air plasmas, the plasma can be pulsed.
The power supply going to be used is able to produce a nano- to microsecond scale pulsed plasma, at variable, kilohertz-range frequencies, with applied voltages of approximately 20 kV. 
This means that the rise time of the wave form controlling the plasma is very fast (nanosecond scale) allowing for a well-controlled, reproducible plasma discharge.
%which is good as it allows for a well-controlled, reproducible plasma discharge.

The configuration of the plasma device also has an impact on how it functions.
Air plasmas have been shown to function in a number of different configurations.
For example, there are configurations where there is a flowing feed gas of air to produce a jet-like plasma where the plasma effluent containing RONS, with or without charged particles, would interact with any biological sample to be treated \cite{Kolb2008cold, Chen2009blood}.
On the other hand, configurations where there is no air flow, but the electrodes ignite the ambient air between them have also been presented \cite{Laroussi2004evaluation}. A variation of this with particular relevance to treatment of biological samples is shown by Fridman \textit{et al}, where the sample to be treated acts as the grounded electrode \cite{Fridman2007floating}.

Due to the nature of this proposed research project, the air plasma device to be designed will utilise a flow of air through the device to carry the plasma effluent towards the sample to be treated.
In particular, it might be best to have the air flow perpendicular to the electric field so that charged particles will not be present in the effluent (as these are hard to measure) and the electric field will not influence the results.
This should result in biological effects that are attributable to measurable components of the plasma effluent.

\subsection*{Aim 2 - Plasma Diagnostics and Numerical Simulations}
In order to quantify the abundance of RONS, techniques such as absorption spectroscopy and laser induced fluorescence (LIF) can be used. 
Due to the high collisionality of air plasmas, absorption spectroscopy techniques, including UV absorption spectroscopy and fourier transform infrared (FTIR) spectroscopy are particularly useful as they are unaffected by processes such as collisional quenching \cite{Niemi2013absolute, Schroter2015atomic}.
However, LIF (and variations such as two-photon LIF) may also be useful as York Plasma Institute (YPI) is unique in having a (TA)LIF laser system with picosecond resolution, thus allowing for measurements of short lifetime species in high collisionality environments, without being too affected by collisional quenching.
%LIF, and variations of, such as two-photon LIF, are more affected by collisional quenching and the lifetime of the species being measured have to be within the resolution of the LIF measurements.
%York Plasma Institute is unique in that the laser system used for (TA)LIF measurements has picosecond resolution, allowing measurements of reactive species in high collisionality environments.
%In order to quantify the abundance of reactive oxygen and nitrogen species, techniques such as absorption spectroscopy, and fourier transform infrared (FTIR) spectroscopy can be used. Techniques such as these are commonly used in atmospheric pressure plasmas, over methods such as (two-photon absorption) laser induced fluorescence ((TA)LIF), as they are unaffected by the high collisionality present in atmospheric pressure plasmas \cite{Niemi2013absolute, Schroter2015atomic}.
Important species to measure include ozone (O$_3$),  hydroxyl radical ($\cdot$OH), atomic oxygen (O), singlet oxygen (\textsuperscript{1}O$_2$), superoxide (O$_2$-), hydrogen peroxide (H$_2$O$_2$),  atomic nitrogen (N), nitric oxide (NO), peryoxynitrite (ONOO-), and other NO$_x$ species \cite{Graves2014low}. 
Absolute densities of these, and additional species of interest, will be investigated both experimentally, and using a 0-dimensional chemical kinetics model. This model is based on a set of chemical reactions and their corresponding reaction rate coefficients. It can also be used to identify the important production and destruction mechanisms for the species of interest.

UV emission by the plasma would also ned to be characterised to determine whether it may be contributing to any biological effects seen.
Treating samples with only UV (with a UV LED), or by placing a UV permeable window between the plasma jet and the sample to allow only UV radiation to reach the sample (while RONS and other particles are blocked), are methods of doing this.

%In addition to RONS, the UV radiation emitted by the plasma would also need to be investigated to determine whether this is also contributing to any observed biological effects of plasma treatment. This has been done before either by treating the sample with only UV radiation with a UV LED, or by placing a UV permeable window between the plasma jet and the sample to allow only UV radiation to reach the sample, while RONS and other particles are blocked.

The temperature of the plasma can be determined using a thermocouple, unless charged particles are present, in which case the second positive system of nitrogen optical emission spectroscopy can be used.
This gives the rotational temperature of nitrogen, which is almost equal to the gas temperature \cite{Twomey2011correlation}.

%In conjunction with the experimental data, results can be compared to global chemistry models.
%Models such as these are useful as they can be used to determine the dominant reactions for the production and destruction of different plasma species as they can be used to look at the decay of different species in the plasma effluent, and by fitting decay curves, the number of decay processes occurring can be estimated.
%There is scope to expand on existing models built for slightly different plasmas, to produce one that is applicable to air plasmas. 



%\subsection*{Aim 3 - Global Chemistry Model}
%There is currently a global model being developed in YPI for helium plasmas with admixtures of water vapour, in order to simulate the plasma chemistry (S. Schr\"{o}ter, unpublished data).
%The aim is to develop this model, and extend it so that it is relevant for air plasmas.
%To work, the model takes into account all the different reactions possible in the plasma and their rate constants (the probability that they will occur), taken from literature.
%Coupled with the plasma parameters such as power, feed gas flow rate and plasma volume, the model can be used to predict the densities of different species, meaning model and experimental data can be compared.
%The model can also be used to determine the dominant reactions for the production and destruction of different plasma species.
%As well as this, it can be used to look at the decay of different species in the plasma effluent, and by fitting decay curves, the number of decay processes occurring can be estimated.


\section*{Cell Assays}

\subsection*{Aim 3 - Plasma cytotoxicity}
To assess the effects of plasma treatment on bacterial (\textit{Staphylococcus aureus}) and eukaryotic cells (keratinocytes), three assays will be performed.
Post plasma treatment, cells can be harvested and split into three groups and assessed simultaneously at multiple time points.
Firstly, a fluorescent exclusion dye would be used to assess membrane integrity and cell death \cite{Kepp2011cell}.
The remaining groups will be used to assess markers of oxidative stress (caused by RONS attack) using fluorescent probes. 
In particular, DNA damage will be assessed via detection of the common DNA oxidation product, 8-OHdG \cite{Valavanidis20098, Dizdaroglu2012oxidatively}, and lipid peroxidation will be examined by detection of lipid peroxide breakdown products such as 4-hydroxynonenal (4-HNE) and malondialdehyde (MDA) \cite{Ayala2014lipid, Joshi2010control, Joshi2011nonthermal}.
The rationale is to correlate oxidative stress with cell death in plasma treated cells over time, and to see if bacterial and eukaryotic cells have different levels of tolerance to oxidative stress before dying.


%The first biological assay would be to determine the rates of cell death and oxidative damage in bacterial cells compared to eukaryotic cells, to see if there is a difference.

%Following LTP treatment of bacterial and eukaryotic cells, three different effects will measured.
%\begin{enumerate}
%\item To measure cytotoxicity, the cells would be stained with a fluorescent exclusion dye. The fluorescence of the cells could then give an indication of the membrane integrity and survival rates of cells \cite{Kepp2011cell}. Ideally, this would be repeated at multiple time points (for both plasma treated and non plasma treated controls) to see if the cytotoxic effects of plasma are long or short lived

%\item DNA damage via detection of 8-OHdG, a DNA oxidation product, widely used as a marker of oxidative stress \cite{Valavanidis20098, Dizdaroglu2012oxidatively}

%\item lipid peroxidation (via the detection of lipid peroxide breakdown products such as 4-hydroxynonenal (4-HNE) and malondialdehyde (MDA) \cite{Ayala2014lipid} using fluorescent probes) can be measured \cite{Joshi2010control, Joshi2011nonthermal}. 

%\end{enumerate}

%The rationale behind this is to look, at the population level, to correlate the DNA death rates with the levels of DNA damage and lipid peroxidation over time. This could indicate whether bacterial and eukaryotic cells can tolerate similar levels of DNA damage and lipid peroxidation before dying. For example, if levels of DNA damage and lipid peroxidation in both types of cells are similar, but the death rate in eukaryotic cells is lower, this may suggest that eukaryotic cells have higher levels of tolerance to these harmful processes. This could potentially equate to some specificity in plasma treatment resulting in bacterial cells being more harmed than eukaryotic cells.

\subsection*{Aim 4 - Promotion of Wound Healing}
Overuse of antibiotics is contributing to the increasing problem of antibiotic resistance, therefore, if wound healing can be accelerated, infection rates should reduce, and therefore reduce the need for antibiotic use.
Here, we aim to investigate the effects of LTP on the speed of wound healing, in particular the initiation of the healing process.
To investigate this, a commercially available \textit{in vitro}, artificial full-thickness skin model, EpiDermFT\texttrademark will be used (MatTek Corporation \cite{MattekWebsite}). 
A wound will be introduced by punch biopsy and the wound healing response will be measured by assessing gene expression over time, via the use of quantitative polymerase chain reaction (qPCR).
Following wounding, the first response by keratinocytes is to become activated to initiate the re-epithelialisation process. This allows them to begin proliferation and migration into the wound space to rebuild the skin barrier.
Under normal, resting circumstances, keratinocytes express keratin (K) 1 and K10, however, upon activation, keratin expression changes to K6, K16 and K17 to signify activation and migration \cite{Pastar2014epithelialization}. 
To allow migration of keratinocytes, they must be released from the basement membrane.
Transcription factors such as Slug are important for this and can, therefore, be a useful marker for investigating the initiation of wound healing \cite{Savagner2005}.
By using markers such as these, we aim to show a time profile of wound healing response in plasma treated and non plasma treated wounds to see if the plasma treated wounds initiate healing faster.
This would be particularly useful in the case of chronic wounds where the healing response has stalled and needs to be restarted, which may, or may, be possible using LTP.

%The inititation of the wound healing response is also important in the case of chronic wounds, as these have often stalled the healing process, therefore, if methods of re-starting the process could be found, it would be beneficial.
%To investigate the potential for LTP to accelerate the wound healing process, particularly at the initiation stage, a commercially available \textit{in vitro}, artificial skin model, namely EpiDermFT\texttrademark (MatTek Corportion \cite{MattekWebsite}), consisting of an epidermis formed by layers of keratinocytes, and a dermis, will be used.
%In this model, a wound can be induced by punch biopsy and then the wound healing response assessed. 
%Following wounding, to begin the process of re-epithelialisation, keratinocytes must first become activated so that they can begin to migrate into the wound space to initiation barrier formation.
%Under normal, resting circumstances, keratinocytes express keratin (K) 1 and K10, however, upon activation, keratin expression changes to K6, K16 and K17 to signify activation and migration \cite{Pastar2014epithelialization}. 
%Further to this, to allow migration of keratinocytes, they must be released from the basement membrane.
%Transcription factors such as Slug are important for this and could therefore be a useful marker for investigating the initiation of wound healing \cite{Savagner2005}.
%By using markers such as these, plasma treated and non plasma treated control samples can be compared to see if the initiation of wound healing begins earlier in the plasma treated tissue. 
%To measure this, methods such as quantitative polymerase chain reaction (qPCR) could be used and gene expression assessed over time determined, in order to understand more about the time profile of wound healing related genes in plasma treated and non plasma treated wounds.

%It has been reported that the speed of the wound healing response can be assessed using fluorescent imaging of the wound over time, by staining for markers of activated keratinocytes and 


%To investigate the effects that LTP has on the wound healing response of human skin cells, an artificial, \textit{in vitro} skin model with an epidermal and dermal layer (formed through layered culture of human donor cells) will be used.
%Such models are commercially available (www.mattek.com), in particular EpiDermFT\texttrademark.
%These models are useful for wound healing investigations as they can be wounded, and the response by the cells measured.
%For example, to begin the process of re-epithelialisation, keratinocytes must first become activated. 
%Under normal circumstances, differentiated keratinocytes express keratin (K) 1 and K10, however, upon activation and migration, this expression swaps to K6, K16 and K17. 
%Therefore it would be of interest to investigate the rate at which keratinocytes begin to express K6, K16 and K17 following plasma treatment, compared to non plasma treated wounds
%
%
%To initiate healing, skin epithelial cells must activate to change from differentiated, stable cells to become proliferative, migrating cells to begin the process of re-epithelialisation.
%By using an \textit{in vitro} artificial skin model \cite{Rasmussen2013classical} and quantitative polymerase chain reaction (qPCR) to investigate the time profile of relevant gene expression post wounding, the aim is to see if plasma can speed up this wound healing response in epithelial cells.
%In particular, the commercially available EpiDermFT model from MatTek Corporation ***CITE???*** has been used previously for investigations into wound healing **CITE***.
%To measure the wound healing response, either the wounded skin could be imaged over time to see if plasma treatment has an effect on the speed that the wound begins to heal. 
%This would be an improvement on the normal scratch wounding models that are often used for looking at wound healing response following plasma treatment, which uses only a monolayer of skin cells, thus losing more of the in vivo situation.
%On the other hand, quantitative polymerase chain reaction (qPCR) to investigate the time profile of relevant gene expression post wounding. 
%Genes of interest would include 
%
%The rationale behind this investigation is to determine whether plasma can accelerate the wound healing process as this would help wounds heal faster, lessening the risk of infection and the need for antibiotics, overuse of which is contributing to antibiotic resistance.
%
%This change can be signified by upregulation of proliferative markers such as Ki-67, a decrease in the proportion of cells in S-phase and an upregulation of $\alpha$5 integrin on epithelial cells during the wound response \cite{Garlick2007engineering}.
%By using an \textit{in vitro} artificial skin model \cite{Rasmussen2013classical} and quantitative polymerase chain reaction (qPCR) to investigate the time profile of relevant gene expression post wounding, the aim is to see if plasma can speed up this wound healing response in epithelial cells.
%If plasma is able to speed up the healing process, it should decrease the chance of infection and the need for antibiotics.

%To determine whether LTP treatment can speed up this process, an \textit{in vitro} artificial skin model can be used \cite{Rasmussen2013classical, Garlick2007engineering}. 
%To investigate this, cellular responses to wounding can be measured with and without plasma treatment. 
%The wounding response results in upregulation of proliferative markers such as Ki-67, a decrease in the proportion of cells in S-phase and an upregulation of $\alpha$5 integrin on epithelial cells during the wound response \cite{Garlick2007engineering}. 
%By looking at the time profile of gene expression relating to these factors, using quantitative polymerase chain reaction (qPCR), it may be possible to detect any differences in plasma treated epithelial cells compared with non treated epithelial cells.
%The rationale behind this is to see if plasma can affect the wound healing response of eukaryotic, host cells in order to speed up the process of healing and, therefore, decrease the chance of infection and the need for antibiotics.

\section*{Contingency Plans}
Due to the difficulties in producing a stable plasma, if this cannot be achieved in a reasonable amount of time, then it would be possible to change to using an argon plasma, supplemented with other gas mixtures, for example oxygen and water vapour. 
Argon is preferable to helium due to lower running costs, therefore would still aim to reduce the cost of plasma treatments.
%nature of air plasmas and the difficulty in producing a stable plasma, if this cannot be achieved in a reasonable amount of time, then it would be possible to change to using an argon plasma, supplemented with other gas mixtures, for example oxygen and water vapour. 
%Argon is preferable to helium when running plasma due to lower costs, therefore would still aim to reduce the cost of plasma treatments.

\section*{Conclusions}
This novelty of this project lies in its investigation into the use of air plasmas in biomedical applications, something which has not been extensively studied previously, particularly with respect to careful characterisation of the plasma produced. Also, the project is designed to give a detailed insight, using novel cell population assays, into the effects of air plasma on both eukaryotic and bacterial cells. This is of critical importance when considering applications such as the acceleration of wound healing due to the involvement of both types of cell. It is necessary to understand the effects that the plasma will have on both the bacterial cells, which we wish to eliminate in order to promote healing, and the eukaryotic cells, which we want to, at least, preserve and hopefully, promote their wound healing response. The programme of work presented above should allow these questions to be addressed.




\bibliographystyle{unsrt}
\bibliography{MyPapers}

\end{document}  